%%% Preamble
\documentclass[paper=a4, fontsize=11pt]{scrartcl}
\usepackage[T1]{fontenc}
\usepackage{fourier}
\usepackage[english]{babel}															

%%% Maketitle metadata
\newcommand{\horrule}[1]{\rule{\linewidth}{#1}} 	% Horizontal rule

\title{
		%\vspace{-1in} 	
		\usefont{OT1}{bch}{b}{n}
		\normalfont \normalsize \textsc{ECSE 211 - Design Principles and Methods} \\ [25pt]
		\horrule{0.5pt} \\[0.4cm]
		\huge Lab 1 - Wall Follower \\
		\horrule{2pt} \\[0.5cm]
}
\author{John Wu \\ Alex Lam}
\date{\today}
 
\begin{document}
 
\maketitle
 
\section{Objective}

The objective of the lab was to create a vehicle that could navigate around around a sequence of cinderblocks making up a wall containing gaps, concave corners, and convex corners, without touching or deviating too far from it.
 
\section{Method}
 
Template code for wall following was modified to:

\begin{itemize}
  \item Avoid getting confused by gaps.
  \item Turn concave corners.
  \item Turn convex corners sharper.
  \item Implement both P and Bang-Bang Controllers for the above requirements.
\end{itemize}
 
\section{Data Analysis}

\begin{itemize}
  \item Did the bang-bang controller keep the robot at a distance Band Centre from the wall?
  \item Why is it expected that the robot will repeatedly oscillate from one side of the band to the other with the bang-bang and p-type controllers?
\end{itemize}

\section{Observation \& Conclusion}
 
What errors did the ultrasonic experience? Were these errors filterable? Does the ultrasonic sensor produce false positives (i.e. the detection of non-existent object), false negatives (i.e. the failure to detect objects), or both

\section{Further Improvements}

What improvements could you make to both the physical or software designs to improve performance of the wall follower? (At least 3 are needed) Are there any other controller types that may have better outcomes than the bang-bang and p-type?

\end{document}
